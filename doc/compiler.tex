\documentclass{article}
\title{Expression compiler and evaluator design}
\begin{document}
\maketitle

\section{Definitions}

\begin{itemize}
  \item An \emph{expression} is a collection of some mathematical calculations to be computed.

  \item \emph{Compilation} (compile-time) is the process of converting a text expression to executable (probably tree-based) form.
  This includes parsing and optimization.

  \item An \emph{evaluation} (run-time) of a (sub)expression computes the value of the expression, taking its inputs into account.

  \item A \emph{top-level evaluation} is one that evaluates the entire expression, not just a subexpression.

  \item A \emph{variable} is an expression whose value may be modified by the client program between top-level evaluations.

  \item Expressions are composed entirely of \emph{operations}, which are functions with some number of inputs, usually other operations.
  Each operation is initially a node in the syntax tree.

  \item The \emph{value} of an operation is the result of executing it.
\end{itemize}

\section{Parser}
The parser should parse expressions into a syntax tree, including affine, rational, and real constants, affine variables, arithmetic, and functions.

Each operation should be tagged with its properties (commutative, associative, (non-, semi-)deterministic, constant, etc.), as well as the properties it desires from its operands (for example \texttt{pow()} would like an integer or rational as its second argument, in that order), sort of a function prototype.

There are several levels of determinism for an operation: let $f(X, n, e)$ be the operation evaluated with input $X$, during the $n^\mathrm{th}$ top-level evaluation, in system environment $e$.
\begin{enumerate}
  \item[D0] Non-deterministic (no guarantees): the operation could have a different value each and every time it is executed.
  Probably not useful, and hence not worth supporting.
  Example: \texttt{rand()}.

  \item[D1] Semi-deterministic ($f(X, n, e_1) = f(X, n, e_2)$): given the same input, the operation returns the same value throughout any single top-level evaluation.
  The operation's value could possible change between two top-level evaluations.
  Example: variables.
  
  \item[D2] Deterministic ($f(X, n_1, e_1) = f(X, n_2, e_2)$): given the same inputs, the value of the operation will always be the same.
  Example: \texttt{sin()}.
  
  \item[D3] Constant ($f(X_1, n_1, e_1) = f(X_2, n_2, e_2)$): the operation has no operands, and always has the same value.
  Example: immediates such as 6.
\end{enumerate}

\subsection{Implementation}
The parsing of expressions into a syntax tree is complete, but the tree does not yet have the necessary tags.

\section{Optimizer}

An expression should be reduced to its most efficient possible form.
This could be executable, or another pass could take the optimizer's output and make an executable form.
Haskell, Lisp, or ML is probably a superior language for the optimization stage.
Integer and rational variables may as well not be permitted.

\subsubsection{Algorithm}
The optimization step should make several passes, performing transformations on the expression tree:
\begin{enumerate}
  \item Propagate determinism levels through operations, starting with terminal nodes:
  \begin{itemize}
    \item A non-constant operation with any non-deterministic operands becomes non-deterministic.
    \item A deterministic operation with any semi-deterministic operands and no non-deterministic operands becomes semi-deterministic.
    \item A deterministic operation with all constant operands becomes constant.
  \end{itemize}
  When finished, there should be no deterministic operations left.

  \item Collapse and evaluate constant subexpressions.

  \item Canonicalize the parse tree.
  Equivalent expressions should have the same canonicalized form, e.g\@. $-x^3 + x^4 \to x^4 - x^3$.

  \item Perform common subexpression elimination.
  Non-deterministic operations may not be collected/eliminated.
  Semi-deterministic and constant operations may be collected/eliminated.

  \item Promote rationals and reals to Affine when their parent
  operators don't have a preference.

  \item Fold remaining rationals and reals (which must all be constants
  at this point) into their parent operators.

  \item Insert copiers in front of constant destination operands and destination operands used by more than one operation; utilize commutative properties to prevent making copies when possible.
  For non-constant operands, keep a list of which operations want to modify it.

  \item For every non-constant operand with a copier above it, give its last modifier direct access.

  \item Make a linear list of all operations to be executed, in order.
  Operations already in the list may be omitted.
\end{enumerate}

\section{Evaluation}
The evaluator uses the list of executable operations and processes them in order.
Most operations modify one of their operands, hence the need for copiers.
First it tells each element in the list to reset itself, then it tells each element to evaluate itself, in order.
The value of the last element is returned.

\end{document}
